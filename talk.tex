\documentclass[aspectratio=169]{beamer}

\makeatletter
\appto\input@path{{libs/awesome-beamer}, {libs/smile}}
\makeatother

\usepackage{fontspec}
\setmonofont[
	Path = ./libs/awesome-beamer/fonts/,
	Scale = .9,
	Extension = .ttf,
	Contextuals=Alternate,
	BoldFont={*-Bold},
	UprightFont={*-Regular},
]{Fira Code}
\renewcommand\setmonofont[2][]{}

\definecolor{forest}{HTML}{1a4301}
\usetheme[english, color, coloraccent=forest]{awesome}

\usepackage{firamath-otf}
\usepackage{pgfplots}
\usepackage{pgfplotstable}
\usepackage{fontawesome}
\usepackage{tikzpingus}
\usepackage{tikzducks}
\usepackage{pdfpc}
\usepackage{amsmath}

\usepgfplotslibrary{dateplot}
\usetikzlibrary{shapes,tikzmark}
\tikzset{pipelinestep/.style={lw,rnd,shape=signal,signal from=west,signal pointer angle=130,minimum width=3cm,minimum height=2cm,draw=black,fill=lightgray!30}}

\def\info#1{\begingroup\color{gray}\scriptsize#1\endgroup}

\newcommand<>{\talknote}[1]{\only#2{\pdfpcnote{- #1}\relax}}

\addbibresource{refs.bib}

\addtobeamertemplate{title page}{}{
\begin{tikzpicture}[overlay, remember picture]
	\node[anchor=south east,outer sep=0pt] at (current page.south east) {\fontsize{3}{3}\selectfont\color{white}This image was generated by AI};
\end{tikzpicture}
}

\pgfplotsset{
	every axis legend/.append style={style={roundednode,fill=accent!10,lcr}},
	every axis plot/.append style={lw,lcr},
}

\title[Renewable Energies: Hydrogen]{Renewable Energies}
\subtitle{Hydrogen}
\author{Lukas Pietzschmann}
\email{lukas.pietzschmann@uni-ulm.de}
\institute{Center for General Scientific Education}
\uni{Ulm University}
\location{Ulm}
\background{background.jpg}
\date\today

\begin{document}
\maketitle

\section{Why there's a Problem}
\newsavebox\sourcesb
\savebox\sourcesb{
	\pgfplotsset{width=\textwidth, height=.7\textheight, compat=1.18}
	\pgfplotstableread[col sep=comma,]{energy.csv}\energy
		\href{https://ourworldindata.org/energy-mix}{\begin{tikzpicture}
	\begin{axis}[axis background/.style={fill=accent!3},
			ylabel=Evergy Consumption (TWh), axis line style={lw,rounded corners},
			xticklabel={\pgfmathint{\tick}\pgfmathresult},
			legend style={
				font=\scriptsize,
				cells={anchor=west},
				legend pos=north west,
			},
			stack plots=y
		]
		\addplot[color=green, smooth] table [x=year,y=other,col sep=comma]{\energy};
		\addlegendentry{Other}
		% \addplot[color=brown, smooth] table [x=year,y=bio,col sep=comma]{\energy};
		% \addlegendentry{Biofuels}
		\addplot[color=yellow, smooth] table [x=year,y=solar,col sep=comma]{\energy};
		\addlegendentry{Solar}
		\addplot[color=blue, smooth] table [x=year,y=wind,col sep=comma]{\energy};
		\addlegendentry{Wind}
		\addplot[color=cyan, smooth] table [x=year,y=hydro,col sep=comma]{\energy};
		\addlegendentry{Hydropower}
		\addplot[color=red, smooth] table [x=year,y=nuclear,col sep=comma]{\energy};
		\addlegendentry{Nuclear}
		\addplot[color=violet, smooth] table [x=year,y=gas,col sep=comma]{\energy};
		\addlegendentry{Gas}
		\addplot[color=teal, smooth] table [x=year,y=oil,col sep=comma]{\energy};
		\addlegendentry{Oil}
		\addplot[color=orange, smooth] table [x=year,y=coal,col sep=comma]{\energy};
		\addlegendentry{Coal}
		% \addplot[color=darkmaroon, smooth] table [x=year,y=bio,col sep=comma]{\energy};
		% \addlegendentry{Biomass}
	\end{axis}
\end{tikzpicture}}}
\begin{frame}
	\frametitle[\cite{tempdata,lenssen2019}]{Global Warming}
	\begin{wide}
		\pgfplotsset{width=\textwidth, height=.7\textheight, compat=1.18}
		\pgfplotstableread[col sep=comma,]{temps.csv}\temps
		\href{https://data.giss.nasa.gov/gistemp/graphs_v4}{\begin{tikzpicture}
			\begin{axis}[axis background/.style={fill=accent!3},
				ylabel=Temperature Anomaly ($^\circ C$), axis line style={lw,rounded corners},
				xticklabel={\pgfmathint{\tick}\pgfmathresult},
				legend style={
					font=\scriptsize,
					cells={anchor=west},
					legend pos=north west,
				},
			]
			\addplot[color=accent, smooth] table [x=Year,y=NoSmoothing,col sep=comma]{\temps};
			\addlegendentry{Annual Mean}
			\addplot[color=red, smooth] table [x=Year,y=Lowess,col sep=comma]{\temps};
			\addlegendentry{Lowess Smoothing}
		\end{axis}
		\end{tikzpicture}}
	\end{wide}
\end{frame}

\newsavebox\sadpingu
\savebox\sadpingu{\tikz{\pingu[small size,eyes=sad,left wing shock,right wing grab]}}
\begin{frame}
	\frametitle[\cite{ritchie2020}]{Energy consumption by source}
	\begin{wide}
		\usebox\sourcesb
	\begin{tikzpicture}[overlay,remember picture]
		\node[anchor=south east,shift={(-6mm,-1mm)}] at (current page.south east) {\scalebox{.8}{\usebox\sadpingu}};
	\end{tikzpicture}
	\begin{itemize}[<+(1)->]
		\item Note that the data shows the global energy consumption
		\item In Germany it's not as bad
	\end{itemize}
	\end{wide}
\end{frame}

\newsavebox\fixpingu
\savebox\fixpingu{\tikz{\pingu[eyes=wink,heart=accent,wings wave, banner=Let's fix the world]}}
\begin{frame}
	\frametitle{How to overcome the problem}
	\onslide<2->Here's the plan:
	\begin{block}
		\begin{enumerate}[<+(2)->]
			\item Produce as much energy from renewable sources as possible
			\item Store it, so we have a backup when the sun doesn't shine
		\end{enumerate}
	\end{block}\onslide<+(2)->
	We ignore 1. for now and focus on 2.
	\onslide<1->\begin{tikzpicture}[overlay,remember picture]
		\node[anchor=south,rotate=-30] at (current page.south west) {\scalebox{.8}{\usebox\fixpingu}};
	\end{tikzpicture}
\end{frame}

\section{How we can store Energy}
\newsavebox\duckforscale
\newsavebox\pumped
\newsavebox\techpingu
\savebox\duckforscale{\tikz\duck[peakedcap=blue];}
\savebox\pumped{\begin{tikzpicture}[decoration={random steps,segment length=3mm,amplitude=1mm}]
	\clip (1,0) rectangle (16,6);
	\pgfmathsetseed{3}
	\coordinate (end) at ([xshift=-2cm]current page.east);
	\coordinate (end2) at ([xshift=2cm]current page.west);
	\coordinate (cliff) at ([shift={(-5mm,7mm)}]end);
	\path[draw=blue,lw,rnd,fill=blue] ([shift={(2mm,6mm)}]current page.east) -- ([shift={(-4mm,6mm)}]end) -- ([shift={(-4mm,-4mm)}]end) -- ([xshift=2mm]current page.east) -- cycle;
	\path[draw=blue,lw,rnd,fill=blue] ([yshift=-3cm]current page.west) -- ([yshift=-3cm]current page.east) -- (current page.south east) -- (current page.south west) -- cycle;
	\path[draw,lw,decorate,rnd,fill=lightgray] (current page.west) -- (end2) -- ([xshift=-2cm]current page.south) -- ([xshift=-2mm]current page.south west) --cycle;
	\path[draw,lw,decorate,rnd,fill=lightgray] ([xshift=2mm]current page.east) -- (end) -- (cliff) -- ([shift={(-2mm,2mm)}]cliff) -- ([xshift=2cm]current page.south) -- ([xshift=2mm]current page.south east) --cycle;
	\node at ([shift={(-5mm,6.5mm)}]current page.east) {\scalebox{.1}{\usebox\duckforscale}};

	\coordinate (A) at ([xshift=-10mm]current page.east);
	\coordinate (B) at ([shift={(25mm,9mm)}]current page.south);
	\path[draw=blue,line width=3mm,rounded corners=0.05mm] (A) |- (B);
	\path[draw,rnd,lw] ([xshift=-1.5mm]A) |- ([yshift=1.5mm]B);
	\path[draw,rnd,lw] ([xshift=1.5mm]A) |- ([yshift=-1.5mm]B);
	\node[roundednode,minimum height=5mm,inner sep=2.5pt,rotate=-30] at (B) {};
	\node[roundednode,minimum width=5mm,inner sep=2.5pt] at (A) {};
\end{tikzpicture}}
\savebox\techpingu{\tikz{\pingu[left wing=wave,laptop left,lightsaber right=blue,headphone=accent,vr-headset,vr-headset hair,heart=accent]}}
\begin{frame}
	\frametitle{Existing energy storage methods}
	\framesubtitle{A small selection}
	\begin{wide}
	\begin{columns}[c]
		\begin{column}{0.5\textwidth}
			\begin{center}
			\onslide<2->\centerline{Pumped Storage Plants}\par\medskip
			\begin{tikzpicture}
				\node[outer sep=0pt,inner sep=0pt] (P) {\scalebox{.4}{\usebox\pumped}};
				\node[roundednode,draw=white,outer sep=0pt,inner sep=0pt,fill=none,fit=(P)] {};
				\node[roundednode,outer sep=0pt,inner sep=0pt,dashed,fill=none,fit=(P)] {};
			\end{tikzpicture}
			\end{center}\medskip\small
			\begin{itemize}[<+(2)->]
				\item[\color{accent}\textbullet] Only a third of the electricity generated is lost
				\item[\color{red}\textbullet] High investment and operating costs
			\end{itemize}
		\end{column}
		\begin{column}{0.5\textwidth}
			\begin{center}
			\onslide<5->\centerline{Lithium-ion Batteries}\par\medskip
			\begin{tikzpicture}
				\node[outer sep=2mm] (P) {\scalebox{.5}{\usebox\techpingu}};
				\node[roundednode,dashed,fill=lightgray!15,fit=(P),node on layer=background] {};
			\end{tikzpicture}
			\end{center}\medskip\small
			\begin{itemize}[<+(3)->]
				\item[\color{accent}\textbullet] Already very common
				\item[\color{red}\textbullet] Lithium is rare and its extraction is complex
			\end{itemize}
		\end{column}
	\end{columns}
	\end{wide}
\end{frame}

\newsavebox\stepab
\newsavebox\stepbb
\newsavebox\stepcb
\savebox\stepab{\tikz\node[pipelinestep,text width=2.2cm, align=center]{Electrolysis};}
\savebox\stepbb{\tikz\node[pipelinestep,text width=2.2cm, align=center]{Storage};}
\savebox\stepcb{\tikz\node[pipelinestep,text width=2.2cm, align=center]{Energy recovery};}
\begin{frame}
	\frametitle{Hydrogen as a source of energy}
	\begin{wide}
	\begin{itemize}[<+(1)->]
		\item Hydrogen is the most common element in the universe~\cite{imagine2021}
		\item On earth, it's primarily found in water
	\end{itemize}\smallskip
	\begin{itemize}[<+(1)->]
		\item Three steps to success:
	\end{itemize}\bigskip\centering
	\begin{tikzpicture}[node distance=2ex]
		\node (A) [visible on=<5->] {\usebox\stepab};
		\node (B) [visible on=<6->,right=of A] {\usebox\stepbb};
		\node (C) [visible on=<7->,right=of B] {\usebox\stepcb};
	\end{tikzpicture}
	\end{wide}
\end{frame}


\newsavebox\morepingu
\savebox\morepingu{\tikz{\pingu[left eye=wink,wings wave,banner=And many more,banner sticks length=1.6cm]}}
\begin{frame}
	\frametitle{Hydrogen as a source of energy}
	\begin{tikzpicture}[overlay,remember picture]
		\node[anchor=north east,outer sep=2mm] at (current page.north east) {\scalebox{.5}{\usebox\stepab}};
	\end{tikzpicture}
	\begin{itemize}[<+(1)->]
		\item We use (green) electricity to split water into hydrogen and oxygen
		\item The electric energy gets transformed into chemical energy
		\item Efficiency of approximately 60\% to 85\%~\cite{milanzi2018}
	\end{itemize}\medskip
	\begin{itemize}[<+(1)->]
		\item But electrolysis isn't the only way to produce hydrogen
	\end{itemize}\medskip
	\begin{wide}
	\begin{tikzpicture}[overlay,remember picture]
		\node[shift={(2.5cm,3.1cm)},visible on=<9->] at (current page.south west) {\scalebox{.6}{\usebox\morepingu}};
	\end{tikzpicture}
	\begin{columns}[c]
		\begin{column}<+(1)->{0.33\textwidth}
			\begin{beamerbox}{green}{}
				Hydrogen from renewable electricity
			\end{beamerbox}
		\end{column}
		\begin{column}<+(1)->{0.33\textwidth}
			\begin{beamerbox}{gray}{}
				Hydrogen from fossil fuels (CO\textsubscript2 released)
			\end{beamerbox}
		\end{column}
		\begin{column}<+(1)->{0.33\textwidth}
			\begin{beamerbox}{blue}{}
				Hydrogen from fossil fuels (CO\textsubscript2 captured)
			\end{beamerbox}
		\end{column}
	\end{columns}
	\end{wide}
\end{frame}

\makeatletter
\newsavebox\storeb
\savebox\storeb{\begin{tikzpicture}[node distance=7mm]
	\node[roundednode,fill=blue!5,minimum width=\beamer@leftsidebar, minimum height=3.5cm] (A) {11\kern2pt000 l};
	\node[roundednode,fill=blue!5,minimum width=\beamer@leftsidebar, minimum height=8mm,below=of A] (B) {24 l};
	\node[roundednode,fill=blue!5,minimum width=\beamer@leftsidebar, minimum height=3mm,below=of B] (C) {14 l};

	\node[below] at (A.south) {\scriptsize Ambient pressure};
	\node[below] at (B.south) {\scriptsize Compressed};
	\node[below] at (C.south) {\scriptsize Liquified};
\end{tikzpicture}}
\makeatother
\begin{frame}
	\frametitle{Hydrogen as a source of energy}
	\begin{tikzpicture}[overlay,remember picture]
		\node[anchor=north east,outer sep=2mm] at (current page.north east) {\scalebox{.5}{\usebox\stepbb}};
		\node[anchor=west,yshift=-5.5mm,xshift=4mm,visible on=<7->] at (current page.west) {\usebox\storeb};
	\end{tikzpicture}
	\begin{itemize}[<+(1)->]
		\item Hydrogen can be safely stored
		\item Conventional methods of storing hydrogen include
			\smallskip\begin{itemize}
				\item Compressed hydrogen\hfill\info{often used in cars}
				\item Liquid hydrogen\hfill\info{used during transportation}
				\item Underground hydrogen storage \hfill\info{can store extremely large amounts}
			\end{itemize}\smallskip
	\end{itemize}
	\begin{itemize}[<+(2)->]
		\item We have the technology to store hydrogen in different capacities for
			different amounts of time
		\item The example on the left illustrates how much space is required by 1kg of
			hydrogen in different states of matter
	\end{itemize}
	\begin{uncoverenv}<10-|handout:2>
	\begin{tikzpicture}[overlay,remember picture]
		\filldraw[darkgray,opacity=0.5] (current page.south west) rectangle (current page.north east);
		\node[squarenode,rnd,fill=white,inner sep=2pt] at (current page) (T) {\includegraphics[width=0.6\paperwidth]{speicherkapazitaet.jpg}};
		\node[yshift=-2mm] at (T.south) {\tiny\citetitle{topler2019}~\parencite{topler2019}};
	\end{tikzpicture}
	\end{uncoverenv}
\end{frame}

\begin{frame}
	\frametitle{Hydrogen as a source of energy}
	\begin{tikzpicture}[overlay,remember picture]
		\node[anchor=north east,outer sep=2mm] at (current page.north east) {\scalebox{.5}{\usebox\stepcb}};
	\end{tikzpicture}
	\begin{itemize}[<+(1)->]
		\item Again, there are different methods to recover energy from hydrogen
		\item Fuel cells are the most common one
		\item In a fuel cell, hydrogen and oxygen react to produce electricity, heat, and water
		\item In this process, roughly 60\% of the energy is converted into
			electricity~\cite{tuv}
	\end{itemize}
\end{frame}

\section{How Hydrogen performs}
\newsavebox\mathpingu
\savebox\mathpingu{\tikz{\pingu[staff left,monocle left,bow tie=accent,small size]}}
\begin{frame}
	\frametitle{The overall efficiency}
	\begin{itemize}
		\item<2-> Let's try to calculate the overall efficiency\\\medskip
			\onslide<3->$\text{Electrolysis} \rightarrow \text{Storing} \rightarrow \text{Energy recovery}$\\
			\onslide<4->$60\% \cdot 98\% \cdot 60\% = 35\%$\hfill\info{lower bound}\\
			\onslide<5->$85\% \cdot \textcolor{gray}{\underbrace{\textcolor{black}{98\%}}_{\text{\clap{Hydrogen stored in salt caverns \cite{ees2022}}}}} \cdot 60\% = 50\%$\hfill\info{upper bound}\\\medskip
		\item<6-> Energy that's lost to heat can easily be used to heat buildings, thus
			increasing the efficiency
		\item<7-> The overall efficiency is worse than that of batteries
	\end{itemize}
	\onslide<1->\begin{tikzpicture}[remember picture,overlay]
		\node[anchor=south west,yshift=-1.3cm,xshift=-1.2cm] at (current page.south west) {\rotatebox{-20}{\usebox\mathpingu}};
	\end{tikzpicture}
\end{frame}


\begin{frame}
	\frametitle[\cite{google}]{Current state}
	\pause
	\begin{wide}
		\pgfplotsset{width=\textwidth, height=.7\textheight, compat=1.18}
		\pgfplotstableread[col sep=comma]{trend.csv}\trend
		\href{https://trends.google.com/trends/explore?date=all&geo=DE&q=Wasserstoff}{\begin{tikzpicture}
			\begin{axis}[date coordinates in=x,axis background/.style={fill=accent!3},
					axis line style={lw,rounded corners},
					xticklabel={\year},
					x tick label style={rotate=45,anchor=east},
					legend style={
						font=\scriptsize,
						cells={anchor=west},
						legend pos=north west,
					},
				]
				\addplot[color=accent,smooth] table [x=date,y=anz]{\trend};
				\addlegendentry{Interest per Month}
			\end{axis}
		\end{tikzpicture}}
	\end{wide}
\end{frame}

\begin{frame}
	\frametitle{Current state}
	\begin{itemize}[<+(1)->]
		\item Our current marked for green hydrogen is still small~\cite{schattauer2022}
		\item But we have concrete plans for the future~\cite{bund}
	\end{itemize}\bigskip
	\begin{itemize}[<+(1)->]\setlength\itemsep{.7em}
		\item[\tikzmark{a}] From 2025 on, we will import 500\kern2pt000 tons of green hydrogen from
			Canada
		\item[\tikzmark{b}] And in 2027/28, our network of hydrogen pipelines will be 1\kern2pt000~km
			long
		\item[\tikzmark{c}] An electrolysis capacity of at least 10 gigawatts is to be built by 2030
		\item[\tikzmark{d}] Germany will need around 18 million tons of green hydrogen by 2050
	\end{itemize}
	\begin{tikzpicture}[remember picture,overlay]
		\coordinate (S) at ([shift={(-2mm,1.5mm)}]pic cs:a);
		\coordinate (E) at ([shift={(-2mm,-1.1cm)}]pic cs:d);
		\draw[-{Latex[round,accent]},lw,visible on=<4->] (S) -- (E);
		\node[roundnode,fill=accent,draw=accent,inner sep=1.5pt,visible on=<4->] at (S) {};
		\node[roundnode,fill=accent,draw=accent,inner sep=1.5pt,visible on=<5->] at ([shift={(-2mm,1.5mm)}]pic cs:b) {};
		\node[roundnode,fill=accent,draw=accent,inner sep=1.5pt,visible on=<6->] at ([shift={(-2mm,1.5mm)}]pic cs:c) {};
		\node[roundnode,fill=accent,draw=accent,inner sep=1.5pt,visible on=<7->] at ([shift={(-2mm,1.5mm)}]pic cs:d) {};
	\end{tikzpicture}
\end{frame}

\begin{frame}
	\frametitle{TLDR}
	\begin{wide}
	\begin{columns}[c]
		\begin{column}{0.45\textwidth}
			\begin{beamerbox}{accent}{Chances}
				\begin{itemize}[<+(1)->]
					\item We can produce hydrogen from renewable sources
					\item Hydrogen has potential applications in various sectors
				\end{itemize}
			\end{beamerbox}
		\end{column}
		\begin{column}{0.45\textwidth}
			\begin{beamerbox}{red}{Challenges}
				\begin{itemize}[<+(1)->]
					\item Currently, hydrogen is rarely green
					\item We don't (yet) have the infrastructure to
						\begin{enumerate}
							\item create,
							\item transport, and
							\item use
						\end{enumerate}
						hydrogen in large enough quantities
				\end{itemize}
			\end{beamerbox}
		\end{column}
	\end{columns}
	\end{wide}
\end{frame}

\section{References}
\begingroup
\defbibheading{bibliography}[\bibname]{}
\let\frametitle\oldft
\begin{frame}[allowframebreaks]
	\frametitle{References}
	\printbibliography
\end{frame}
\endgroup
\end{document}
